\chapter*{Introducci\'on}
\addcontentsline{toc}{chapter}{Introducci\'on}

\section*{Motivación}

Para el actual documento, las razones que motivaron al alumno presente a realizar la investigación de su tesis son la de comenzar a 
entrar en una rama que en los últimos años ha tomado bastante importancia en la informática, conocida como la {\textit{Bioinformática}} \cite{bioinformatica}, la cual aplica las tecnologías computacionales contemporáneas a datos biológicos que pueden pertenecer a estructuras como ADN, proteínas, entre otras estructuras biológicas complejas y los cuales están a la mano del ser humano como archivos de cadenas de secuencias (en varios formatos) y que pueden ser usados a voluntad.

En el caso puntual de esta memoria, se trabajarán con proteínas (compuestas de combinaciones de 20 aminoácidos) que están distribuidas en un {\textit{dataset}} cuyo formato del archivo está en \textbf{.fasta}, donde cada cadena de polipéptido está compuesto por un ID o código identificador, su nombre taxonómico y su posterior secuencia de aminoácidos. 

Para este tipo de biomolécula, se han realizado numerosos análisis \cite{searching, array} en bases de datos de secuencias con respecto a la cantidad de aminoácidos que se encuentran en total en este tipo de estructuras, distribuyéndolos según su tamaño o para determinar cuál es el aminoácido que más aparece en este tipo de archivos; también según el año de descubrimiento de las proteínas o el tipo de proteína. Todo esto usando fuentes como UniProt y EROP-Moscow, bases de datos de polipéptidos muy importantes que han recabado mucha información sobre este tipo de estructura. 

Como una derivación directa, existe una variabilidad casi infinita de combinaciones llamadas residuos (fragmentos) de aminoácidos ({\textit{amino acid residues}} o AAR de manera simplificada en inglés) que se determinan según su tamaño $k$ y por las posibles opciones a obtener. Por consiguiente buscar y sumergirse en ese universo de posibilidades de encontrar ciertos residuos considerando la cantidad de proteínas que existen en la actualidad es una motivación muy grande, ya que implica un gran desafío el de meterse a buscar ``una aguja en un pajar'' si se mira desde un punto más coloquial.