\documentclass[letterpaper, 12pt, notitlepage]{report}
\usepackage[plainpages=false, colorlinks, urlcolor=black, citecolor=black,
	linkcolor=black]{hyperref}
\usepackage{graphicx}

%A LaTeX package which provides macros for the graphical
%representation of the keys on a computer keyboard.
\usepackage{keystroke}

%http://ctan.dcc.uchile.cl/help/Catalogue/entries/floatrow.html
%\usepackage{floatrow}

\usepackage{longtable}

\usepackage[utf8]{inputenc}
\usepackage{txfonts}
\usepackage{listings}
\usepackage{multirow}
\usepackage{float}
\usepackage[polutonikogreek,english,spanish]{babel}
\usepackage{epsfig}
\usepackage{sty/utfsm_tesis}

%Include Table of Contents as the first entry in TOC
\usepackage{sty/xtocinc}
\usepackage[nottoc]{tocbibind}

%\usepackage[dvips]{epsfig}
%\usepackage{psfig}

\usepackage{caption}
\usepackage{subcaption}

%% Margenes segun Normas
%% Ancho Legal 21,59cm /  8,5in
%% Alto  Legal 33,02cm / 13,0in
%\paperheight    27.81cm % alto letter
%\paperwidth     21.59cm % ancho
%\hoffset        -1.0in % Seteo a 0 el margen izquierdo
%\voffset        -1.0in % Seteo a 0 el margen superior
%\oddsidemargin  3.80cm % Margen izquierdo (pag. impar)
%% \evensidemargin 2.55cm % Margen izquierdo (pag. par) Acrobat Winkk
%\evensidemargin 2.59cm % Margen izquierdo (pag. par)
%\topmargin      1.00cm % Margen superior
%\headheight     5.00mm % Ancho encabezado
%\headsep        8.00mm % Separacion encabezado-cuerpo
%\textheight     23.5cm % Alto cuerpo
%\textwidth      15.2cm % Ancho cuerpo
%\footskip       1.30cm % Separacion piepag-cuerpo
\parindent      0em
\parskip        2ex

% Fuzz -------------------------------------------------------------------
\hfuzz2pt

% \oddsidemargin	 0cm	% Ancho Legal 21,59cm
% \evensidemargin 0.5cm	% Alto	Legal 35,56cm
% \textwidth	 16.5cm
% \topmargin	  -1.5cm
% \textheight	 22cm

\newlength{\defbaselineskip}
\setlength{\defbaselineskip}{\baselineskip}

\newcommand{\setlinespacing}[1]%
	   {\setlength{\baselineskip}{#1 \defbaselineskip}}
\newcommand{\doublespacing}{\setlength{\baselineskip}%
			   {1.3 \defbaselineskip}}
\newcommand{\singlespacing}{\setlength{\baselineskip}{\defbaselineskip}}

\lstloadlanguages{C++, sh, IDL, make}
\lstset{basicstyle=\small\sffamily, commentstyle=\slshape,
        numbers=left, numberstyle=\tiny, numbersep=10pt,
        extendedchars, frame=lines,
        floatplacement=ht, captionpos=b,
        defaultdialect=[CORBA]IDL}



%descomentar para poner fecha distinta a fecha de compilacion
\copyrightyear{2017} \submitdate{Octubre 2017}
\convocation{Octubre}{2017}

\title{Algoritmo para el cálculo de fragmentos de proteínas en los organismos secuenciados}
\author{Felipe Nicolás Araya Barrera}

\begin{document}

\selectlanguage{spanish}

\profguia{Lioubov Dombrovskaia}
\profcorr{Diego Arroyuelo Billiardi}

% Archivo de Agradecimientos
\ack{include/acknowledgements}

% Incluir Resumen
\resumenesp{include/resumen}

% Incluir Abstract
\resumening{include/abstract}

% Incluir Abreviaciones
\abreviaciones{include/glossary}

\beforepreface
\afterpreface

%\numberwithin{equation}{chapter}

% Incluir introduccion
\chapter*{Introducci\'on}
\addcontentsline{toc}{chapter}{Introducci\'on}

Este trabajo de se encuentra en una rama que en los últimos años ha tomado bastante importancia en la informática, conocida como la {\textit{Bioinformática}} \cite{bioinformatica}, la cual aplica las tecnologías computacionales contemporáneas a estructuras como ADN, proteínas, entre otras estructuras biológicas complejas y los cuales están a la mano del ser humano como archivos de cadenas de secuencias.

Las proteínas (compuestas de combinaciones de 20 aminoácidos) han sido objetos de estudios \cite{searching, array} con respecto a la cantidad de aminoácidos que los conforman, según el año de descubrimiento de las proteínas o el tipo de proteína.

Las combinaciones llamadas residuos (fragmentos) de aminoácidos ({\textit{amino acid residues}} o AAR de manera abreviada en inglés) se determinan según su largo $k$. Buscar ciertos residuos dada la cantidad de proteínas que existen en la actualidad es un desafío muy grande.

%inicio desarrollo
%archivo definicion del problema
\chapter{Definici\'on del Problema}

\section{Definición}

Las proteínas desempeñan un papel fundamental para la vida y son las biomoléculas más versátiles y diversas, las cuales realizan una enorme cantidad de funciones diferentes, tales como enzimáticas, estructurales, inmunológicas, entre otras. Para muchos biólogos y científicos especializados en este tipo de biomolécula resulta crucial investigar sobre tales propiedades anteriormente mencionadas, por lo tanto para ellos es necesario saber o conocer la composición básica de cada una de las proteínas en base a su elemento básico, conocido como el aminoácido (existen 20 diferentes en total). Todas las proteínas se componen de aminoácidos, entregando así una cantidad inmensa de polipéptidos que existen y que van apareciendo gracias al trabajo de investigaciones y proyectos que hacen los encargados de este asunto.

Estas proteínas aparecen registradas en base de datos (como UniProt, Genbank) con su secuencia, su nombre taxonómico y un código en clave también conocido como ID, las cuales están disponibles en archivos con formato \textbf{.fasta} (que pueden abrise usando un editor de texto básico) de la siguiente forma:

\newpage

\begin{figure}[ht]
    \centering
    \includegraphics[width=0.8\textwidth]{./images/secuencias.png}
    \caption{Secuencias de varias proteínas en formato \textbf{.fasta}}
    \label{fig:image4}
\end{figure}

Una de las principales tareas en la investigación de proteínas consiste es buscar residuos de aminoácidos (secuencias de aminoácidos o \textit{aminoacid residues (AAR)} en inglés) en el conjunto universo de las polipéptidos para poder localizar cuáles son los residuos de ciertos tamaños que más aparecen en las bases de datos. El problema principal radica en buscar cuántas veces aparece cierto residuo de una base de datos con tamaños bastante considerables (por ejemplo, buscar una determinada subsecuencia de 2 aminoácidos en una base de datos de 500 GB o superior) sin saber si este residuo está presente o no en aquella base de datos, esto podría provocar gastos innecesarios de tiempo a la hora de realizar este tipo de búsqueda, por consiguiente se desea ocupar el menor tiempo posible para realizar esta tarea.

\section{Objetivos}

Lo que se pretende realizar para esta memoria consiste en obtener de un conjunto predeterminado de proteínas el número máximo de fragmentos de péptidos (AAR) que existen asociado a un valor $k$ determinado, donde $k$ se ubicará en el intervalo entre 2 hasta 50, y en base a aquello obtener y determinar para cada $k$ cuáles son los fragmentos de aminóácidos que más se repiten para posteriormente realizar un análisis tanto matemático y biológico de los resultados obtenidos. Para realizar esta tarea se usará como \textit{dataset} la base de datos de proteínas de SwissProt (550100 proteínas) y de TrEMBL (88032926 proteínas).

A partir de esto las implementaciones de código para trabajar estos archivos serán realizados en los lenguajes de programación {\textbf{C++}} y {\textbf{Python}}, con diferentes finalidades que serán explicados a medida que se avance en el documento.
%archivo estado del arte
\chapter{Estado del Arte}

\section{Información previa a considerar}

Para entrar de lleno en la tema, es necesario conocer de antemano varios aspectos básicos de la biología.

\subsection{Biomoléculas}

Cada vez que se habla de la biología, este concepto se relaciona directamente con la ciencia que estudia a los seres vivos. Ahora bien, las estructuras o compuestos que constituyen una parte esencial de los seres vivos son conocidas como \textbf{biomoléculas}. Estas biomoléculas están principalmente constituidas por elementos químicos como el carbono (C), hidrógeno (H), oxígeno (O), nitrógeno (N), fósforo (P) y azufre (S) \cite{biomolecula} y se pueden clasificar en biomoléculas inorgánicas, que se encuentran tanto en seres vivos como en los cuerpos inertes, no obstante son imprescindibles para la vida; y las biomoléculas orgánicas, que son sintetizadas por los seres vivos y tienen una estructura con base en carbono. Estas biomoléculas orgánicas se pueden separar en 4 grandes grupos:
\begin{enumerate}
\item Glúcidos (hidratos de carbono o carbohidratos): son la fuente de energía primaria que utilizan los seres vivos para realizar sus funciones vitales. Los ejemplos más conocidos son la glucosa, el almidón y el glucógeno.
\item Lípidos: conforman el principal almacén de energía de los animales y desempeñan funciones reguladores de enzimas y hormonas.
\item Ácidos nucleicos: El ácido desoxirribonucleico y el ácido ribonucleico, mayormente conocidos como ADN (DNA) y ARN (RNA y sus derivados) desarrollan posiblemente la función más importante para la vida: contener, de manera codificada, las instrucciones necesarias para el desarrollo y funcionamiento de la célula. El ADN tiene la capacidad de replicarse, transmitiendo así dichas instrucciones a las células hijas que heredarán la información.
\item Proteínas: poseen la mayor diversidad de funciones que realizan en los seres vivos; prácticamente todos los procesos biológicos dependen de su presencia y/o actividad. Son proteínas casi todas las enzimas, catalizadores de reacciones metabólicas, hemoglobina, anticuerpos, entre otros. Su unidad base es el {\it{aminoácido}}, por el cual se van formando los péptidos según la cantidad de unidades bases enlazadas.
\end{enumerate}

Dentro de estas biomoléculas, el análisis detallado de los carbohidratos y los lípidos depende en demasía de su estructura química (elementos químicos asociados y tipo de enlaces entre ellos), por lo mismo es una materia más ligada a los químicos (ver Figura 1.1). 

\begin{figure}[H] 

\begin{subfigure}{0.5\textwidth}
\includegraphics[width=0.9\linewidth, height=5cm]{./images/glucidoejemplos} 
\caption{Glúcidos (glucosa y fructosa)}
\label{fig:subim1}
\end{subfigure}
\begin{subfigure}{0.4\textwidth}
\includegraphics[width=1\linewidth, height=5cm]{./images/lipidosejemplos}
\caption{Lípidos (colesterol y un ácido graso)}
\label{fig:subim2}
\end{subfigure}
 
\caption{Estructura química de los carbohidratos y lípidos.}
\label{fig:image1}
\end{figure}

Sin embargo, el ADN y los polipéptidos poseen unidades base que pueden ser codificadas como letras, por consiguiente pueden ser secuenciados como {\it{cadenas de strings}} y en donde los avances computacionales y la evolución informática toman una importante relevancia (ver Figura 1.2).

\begin{figure}[H]

\begin{subfigure}{0.5\textwidth}
\includegraphics[width=1\linewidth, height=2cm]{./images/adnejemplo}
\caption{Cadena de ADN aleatoria}
\label{fig:subim4}
\end{subfigure}
\begin{subfigure}{0.4\textwidth}
\includegraphics[width=1\linewidth, height=5cm]{./images/cadena} 
\caption{Cadenas $\alpha$ y $\beta$ de hemoglobina bovina}
\label{fig:subim3}
\end{subfigure}
 
\caption{Biomoléculas de ADN y péptidos llevadas a cadenas de strings.}
\label{fig:image2}
\end{figure}

Con respecto a estas 2 últimas estructuras, la diferencia visual más notoria radica en la cantidad de diferentes letras (strings) que las componen, para el ADN son 4 [8] y son denominadas \textbf{bases nitrogenadas} que son las siguientes:

\begin{enumerate}
\item Adenina
\item Timina
\item Citosina
\item Guanina
\end{enumerate}

Para las proteínas, su elemento básico, como ya se mencionó anteriormente es el \textbf{aminoácido}, pero ahora se adentrará en más detalle sobre esta molécula.

\subsection{Aminoácidos}

Los aminoácidos tienen diferentes funciones en el organismo \cite{amino} pero ante todo sirven como \textbf{las unidades básicas de los péptidos y de las proteínas.} A nivel orgánico el aminoácido es una molécula compuesta con un grupo amino (-NH2) y un grupo carboxilo (-COOH) y que pueden tener distintas distribuciones. Para el caso de los que componen las proteínas se consideran como alfa-aminoácidos:

\begin{figure}[h]
    \centering
    \includegraphics[width=0.4\textwidth]{./images/aminoacido}
    \caption{Estructura general de un alfa-aminoácido}
    \label{fig:image3}
\end{figure}

En la imagen anterior se puede identificar el carbono central (alfa) unido al grupo carboxilo (rojo), grupo amino (verde), un hidrógeno (imagen superior color negro) y el grupo radical (azul) o R. Este grupo radical es el que determina la identidad y las propiedades de cada uno de los diferentes aminoácidos.

El primer aminoácido fue descubierto a principios del siglo XIX, y a partir de ese entonces hasta la actualidad son miles los aminoácidos que han sido descubiertos, pero solo 20 se consideran como los componentes esenciales para las proteínas (y los que se considerarán como parte de esta memoria) que se presentarán a continuación en conjunto con su respectiva abreviación utilizada en las cadenas de proteínas de los archivos FASTA:

\begin{enumerate}
\item Alanina - A
\item Cisteína - C
\item Ácido aspártico - D
\item Ácido glutámico - E
\item Fenilalalina - F
\item Glicina - G
\item Histidina - H
\item Isoleucina - I
\item Lisina - K
\item Leucina - L
\item Metionina - M
\item Aspargarina - N
\item Prolina - P
\item Glutamina - Q
\item Arginina - R
\item Serina - S
\item Treonina - T
\item Valina - V
\item Triptófano - W
\item Tirosina - Y
\end{enumerate}

Existen otras abreviaturas en las cadenas como B, X o J, pero para el alcance de esta memoria no serán considerados como objeto de estudio y análisis posterior.

A partir de este pequeño elemento se forman las macromoléculas que se identifican según la cantidad de aminoácidos (a partir de ahora se mencionarán como aa.) que lo compongan:

\begin{table}[H]
\centering
\label{my-label1}
\begin{tabular}{|c|c|}
\hline
Tamaño & \multicolumn{1}{c|}{Tipo de estructura}  \\ \hline
2 aa.     & Dipéptido        \\
3 aa.     & Tripéptido                         \\
Entre 2 y 8 aa.      &      Oligopéptido                             \\
Menos de 100 aa.      &   Péptido       \\
Mayor o igual de 100 aa.   &   Proteína o polipéptido            \\ \hline
\end{tabular}
\caption{Identificación de macromoléculas según cantidad de aminoácidos}
\end{table}

\section{Secuencias de proteínas}

Desde el momento en que se descubrieron los elementos componentes del ADN y las proteínas, se han investigado sobre las posibles combinaciones que se pueden encontrar entre las bases que los conforman y en que cantidad se encuentran. Para el ADN y sus 4 elementos básicos existen millones de seres vivos, parásitos, virus, protozoos y entre otros que se definen por su código genético, por lo cual encontrar los diversos residuos de bases según un determinado tamaño. En el caso puntual para la finalidad de este escrito y según lo mencionado por \cite{zamyatnin1}, es posible estudiar de manera teórica y con fórmulas matemáticas la cantidad máxima de fragmentos que puede formar una proteína. Considerando como base que el número posible de estructuras peptídicas naturales $P$ están compuestas de diferentes residuos de aminoácidos (incluyendo repeticiones en cadenas de aminoácidos) sigue la siguiente fórmula:

\begin{equation}
P=A^{n}
\end{equation}

Donde $A$ es el número de diferentes aminoácidos existentes, y $n$ es la cantidad de aminácidos correspondientes a la estructura estudiada. Por lo mismo y siguiendo esta fórmula (considerando $A=20$) la cantidad de diferentes combinaciones péptidos de tamaño k que se pueden obtener se aprecian en la siguiente tabla:

\begin{table}[H]
\centering
\label{my-label2}
\begin{tabular}{|c|c|}
\hline
Tamaño péptido (k) & \multicolumn{1}{c|}{Combinaciones posibles ($A^{k}$)}  \\ \hline
2 aa.     & 400        \\
3 aa.     & 8000                         \\
4 aa.      &      160000                             \\
5 aa.      &   3200000       \\
10 aa.      &   1.024$\times 10^{13}$       \\
20 aa.      &   1.049$\times 10^{26}$       \\
50 aa.   &     1.126$\times 10^{65}$   \\ \hline
\end{tabular}
\caption{Combinaciones posibles a obtener según el tamaño del péptido}
\end{table}

Según lo observado en esta tabla se identificar que a medida que el valor de k va en aumento, las posibles combinaciones que se pueden obtener de fragmentos de proteínas pueden llegar a tener valores inimaginables para el ser humano corriente; no obstante, no todas estas estructuras existen o son capaces de ser encontradas en la naturaleza \cite{array}, aun así la diversidad de la búsqueda de estos residuos sigue siendo gigantesca, y por ende difícil de solucionar, y este será uno de los problemas que se intentará solucionar en esta memoria.

Ahora bien, para un polipéptido de tamaño $n$ aminoácidos, el máximo número posible de fragmentos de tamaño $k$ que teóricamente se podrían obtener (considerando las posibles repeticiones de fragmentos) es descrita mediante la siguiente expresión:

\begin{equation}
N_{k}^{teorica}=n+k-1
\end{equation}

Por consecuencia, el máximo número posible de fragmentos (incluye posibles repeticiones) que teóricamente se pueden obtener para una molécula de tamaño $n$, partiendo desde $k=2$ (dipéptidos) hasta $k=n-1$, viene dado por:

\begin{equation}
N_{suma}^{teorica}=\sum_{2}^{n-1} \frac{k(k-1)}{2}-1
\end{equation}

Mediante estas fórmulas, se han calculado la cantidad de posibles fragmentos que se pueden obtener en diferentes oligopéptidos y proteínas:

\begin{table}[H]
\centering
\label{my-label3}
\begin{tabular}{|c|l|r|r|}
\hline
Número & \multicolumn{1}{c|}{Oligopéptido/Proteína} & \multicolumn{1}{c|}{$n$} & \multicolumn{1}{c|}{$N_{suma}^{teorica}$} \\ \hline
1      & Encefalina (varios tipos biológicos)       & 5                        & 9                     \\
2      & Bradiquinina (mamíferos)                   & 9                        & 35                    \\
3      & ACTH (humanos)                             & 39                       & 740                   \\
4      & Cadena $\alpha$ hemoglobina (humanos)          & 141                      & 9869                  \\
5      & Cadena $\beta$ hemoglobina (humanos)          & 146                      & 10584                 \\ \hline
\end{tabular}
\caption{Número máximo posible de fragmentos que se pueden formar en 5 proteínas}
\end{table}

Considerando que para un polipéptido de largo $n$ aminoácidos, si este valor de $n$ es muy alto, se puede obtener una cantidad muy alta de fragmentos de dipéptidos, pero muchos de estos dipéptidos se pueden repetir varias veces en la cadena, por lo tanto, cuando se desea obtener {\bf{el máximo número de fragmentos diferentes}} asociado a un valor $k$ determinado, este puede tener un valor muy bajo en comparación con la cantidad total de fragmentos obtenidos. Por medio de las fórmulas descritas anteriormente, se puede obtener el número máximo de diferentes fragmentos (o fragmentos esperables) asociado a un tamaño $k$:

\begin{equation}
N_{k}^{diff}=N_{k}^{teorica}- R_{k}
\end{equation}

Este valor $R_{k}$, se obtiene introduciendo nuevos parámetros $i$ (que es el número de estructuras idénticas para determinado $k$) y $m$ (el número de diferentes estructuras para el determinado $k$):

\begin{equation}
R_{k}=\sum_{1}^{m}(i-1)
\end{equation}

Por lo tanto, el número máximo de fragmentos diferentes que se pueden obtener en una proteína sigue la siguiente fórmula:

\begin{equation}
N_{suma}^{diff}=\Bigg[\sum_{2}^{n-1} \frac{k(k-1)}{2}-1\Bigg]- \sum_{2}^{n-1}\Bigg[\sum_{1}^{m}(i-1)\Bigg]
\end{equation}
\\
Tomando la información de la base de datos de oligopéptidos EROP-Moscow, para mostrar los valores obtenidos con estas fórmulas, se usará como ejemplo la caseína bovina (proteína proveniente de la vaca). Esta proteína se compone de 4 subunidades, $\alpha - s1$, $\alpha -s2$, $\beta$ y $\kappa$. La siguiente tabla muestra las cantidades teóricas y diferentes de fragmentos obtenidos como dipéptidos y sus sumas totales:

\begin{table}[H]
\centering
\label{my-label4}
\begin{tabular}{|c|l|c|c|c|c|c|}
\hline
Número & \multicolumn{1}{c|}{Caseína bovina (subunidad)} & $n$ & $N_{2}^{teorica}$ & \multicolumn{1}{l|}{$N_{2}^{diff}$} & \multicolumn{1}{l|}{$N_{suma}^{teorica}$} & \multicolumn{1}{l|}{$N_{suma}^{diff}$} \\ \hline
1      & $\alpha - s1$                                   & 199 & 198               & 134                                 & 19700                                     & 19621                                  \\
2      & $\alpha - s2$                                   & 207 & 206               & 131                                 & 21320                                     & 21216                                  \\
3      & $\beta$                                         & 209 & 208               & 124                                 & 21735                                     & 21641                                  \\
4      & $\kappa$                                        & 169 & 168               & 118                                 & 14195                                     & 14138                                  \\
5      & $\alpha - s1 + \alpha - s2 + \beta + \kappa$    & 784 & 780               & 260                                 & 76950                                     & 76304                                  \\ \hline
\end{tabular}
\caption{Número máximo posible de fragmentos que se pueden obtener en una proteína de caseína bovina.}
\end{table}

Se puede identificar que para los fragmentos de dipéptidos, la cantidad de diferentes fragmentos es bastante menor que la cantidad total de fragmentos obtenidos para las 4 subunidades, pero aún así la cantidad de fragmentos diferentes totales obtenidos es prácticamente la misma que la cantidad de fragmentos totales sin diferenciar. Esto es notorio ya que si $k$ va en progresivo aumento, el universo combinatorio de posibles fragmentos formados se acorta drásticamente, lo que también favorece a la baja formación de fragmentos que se repiten.\\


\section{Técnicas utilizadas en el problema}

En \cite{searching} se menciona que buscar secuencias de proteínas en un predeterminado archivo (puede ser de texto o .fasta) es una tarea muy compleja, ya que se formaría un escenario similar al buscar una {\textit{aguja en un pajar}} y recorrer millones de secuencias en cada búsqueda no sería lo más convieniente considerando que la cantidad de proteínas que existen el día de hoy son muchas, por consiguiente una herramienta recomendable sería preprocesar la base de datos de proteínas con alguna técnica conocida o implementada. A continuación se hablará de forma general de algunos algoritmos conocidos que han tratado este problema.

\subsection{Algoritmo de fuerza bruta}

Este es el algoritmo más simple posible, ya que dado un texto de tamaño \textit{n} se revisan todas las posiciones posibles de un patrón de tamaño \textit{k < n} desde el comienzo hasta el final del texto (izquierda a derecha). Para el caso puntual de los archivos .fasta de las proteínas es necesario extraer únicamente las cadenas de secuencias respectivas y adjuntarlas línea a línea (de esa forma es más fácil trabajarlas). Luego, y siguiendo la presunción matemática del número máximo de fragmentos de tamaño \textit{k} cada una de las cadenas se revisa desde la posición 0 hasta la posición \textit{n-1} (el valor de \textit{n} varía según el largo de la cadena de cada proteína) yendo de caracter a caracter un total de \textit{n+k-1} veces para cada secuencia:

\subsection{Algoritmos de búsqueda de strings}

Esta clase de algoritmos (en inglés conocidos como \textit{string searching algorithm}) tratan de localizar si uno o varios strings solicitados (que también se llaman patrones) aparecen en un string más largo o simplemente un texto como tal, considerando el alfabeto que por el cual está compuesto el string de destino o texto. En la mayoría de las ocasiones este alfabeto ($\Sigma$) es el que determina el rendimiento de determinado algoritmo (una variación de este alfabeto puede ayudar o perjudicar la eficiencia de un algoritmo en particular), como también el texto a analizar.
Una clasificación básica de estos algoritmos se puede realizar según la cantidad de strings a encontrar:
\begin{enumerate}
\item Algoritmos de búsqueda de un único patrón (\textit{Single pattern algorithms})
\item Algoritmos de búsqueda de múltiples patrones (\textit{Multiple pattern algorithm})
\end{enumerate}

\subsubsection{Algoritmos de búsqueda de un único patrón}
Como lo dice el mismo título, esta clase de algoritmo de búsqueda consiste en encontrar las ocurrencias de un determinado patrón 
%archivo propuesta
\chapter{Implementación}

\section{Propuesta considerada}

Examinando las técnicas anteriormente revisadas, se llega al punto de que lo más factible es trabajar con las cadenas de secuencias utilizando un arreglo que las encadene una a una. Recordando los objetivos que se tienen para esta memoria, estas son:

\begin{enumerate}
\item Obtener la cantidad total de diferentes residuos de aminoácidos de tamaño $k =$ 1 hasta 50 que existen para las bases de datos de UniProt-SwissProt y UniProt-TrEMBL.
\item Encontrar para cada caso anterior cuáles son los residuos de aminoácidos que más se repiten.
\end{enumerate}

Para realizar la primera tarea, será necesario construir un \textit{suffix array} el cual será la base del arreglo LCP para realizar este objetivo. Considerando un ejemplo sencillo como la palabra BANANA\$:

\begin{table}[H]
	\centering
	\begin{tabular}{c l}
		\textit{\textbf{SA[]}} & \textit{\textbf{sufijo}}\\
		6 & \$\\
		5 & A\$\\
		3 & ANA\$\\
		1 & ANANA\$\\
		0 & BANANA\$\\
		4 & NA\$\\
		2 & NANA\$\\
	\end{tabular}
\end{table}

Se puede apreciar que los números asociados a cada sufijo ya están ordenados como si fuera un arreglo de sufijos. Es posible obtener la cantidad total de diferentes substrings que componen esta palabra utilizando el arreglo LCP de la siguiente forma. Introduciendo los 2 siguientes conceptos:

{\it{length}}('X') = Largo de caracteres de la palabra 'X'.\\
{\it{LCP}}('Y','Z') = Prefijo más largo en común ({\it{Longest Common Prefix}}) entre los substrings 'Y' y 'Z'.

Y partiendo según el orden alfabético dado anteriormente, se hace el siguiente ejercicio:

Largo primer sufijo ordenado ('\$') = 1 = $var$\\
Comienzo de pares de sufijos:
\begin{enumerate}
	\item ('\$','A\$'): $var +=$ {\it{length}}('\$A') - {\it{LCP}}('\$','\$A')\\
	$var=var+2-0 =>  1+2=3$

	\item ('A\$','ANA\$'): $var +=$ {\it{length}}('ANA\$') - {\it{LCP}}('A\$','ANA\$')\\
	$var=var+4-1 =>  3+3=6$
	
	\item ('ANA\$','ANANA\$'): $var +=$ {\it{length}}('ANANA\$') - {\it{LCP}}('ANA\$','ANANA\$')\\
	$var=var+6-3 =>  6+3=9$
	
	\item ('ANANA\$','BANANA\$'): $var +=$ {\it{length}}('BANANA\$') - {\it{LCP}}('ANANA\$','BANANA\$')\\
	$var=var+7-0 =>  9+7=16$
	
	\item ('BANANA\$','NA\$'): $var +=$ {\it{length}}('NA\$') - {\it{LCP}}('BANANA\$','NA\$')\\
	$var=var+3-0 =>  16+3=19$
	
	\item ('NA\$','NANA\$'): $var +=$ {\it{length}}('NANA\$') - {\it{LCP}}('NA\$','NANA\$')\\
	$var=var+5-2 =>  19+3=22$
	
\end{enumerate}

Cantidad de diferentes substrings que hay en BANANA\$: 22.

Lo que se hace en este caso es crear una variable y guardar la longitud del primer sufijo del SA (en este caso \$) para luego realizar una comparación entre los sufijos consecutivos $i$ y $j$ adicionando en cada caso a la variable las longitudes respectivas de los sufijos $j$, y además \textbf{se le resta el prefijo común más largo entre estos sufijos consecutivos}:

\begin{table}[H]
	\centering
	\label{propuesta-1}
	\begin{tabular}{c c l}
		\textit{\textbf{SA[]}} & \textit{\textbf{LCP[]}} &\textit{\textbf{sufijo}}\\
		6 & 0 & \$\\
		5 & 0 & A\$\\
		3 & 1 & ANA\$\\
		1 & 3 & ANANA\$\\
		0 & 0 & BANANA\$\\
		4 & 0 & NA\$\\
		2 & 2 & NANA\$\\
	\end{tabular}
\caption{SA y arreglo LCP de la palabra BANANA\$}
\end{table}

Entonces, particularizando el problema, ¿cómo sería posible obtener la cantidad total de diferentes substrings de un determinado tamaño? La clave está en el \textbf{arreglo LCP} obtenido, el cual se puede utilizar desde 2 perspectivas para realizar esta tarea.

\subsection{Restando la cantidad máxima posible de diferentes substrings}

Primero que todo hay que considerar la cantidad potencial máxima de diferentes substrings de tamaño $k$ que se pueden obtener (la fórmula es $n-k+1$ donde $n$ es el largo de la palabra) y luego se recorre el arreglo LCP utilizando el valor de $k$ como un comparador.
Por ejemplo, de la palabra BANANA\$ a simple vista se sabe que los diferentes substrings de tamaño 1 que se encuentran son 4, que son A, B, N y \$. Usando la fórmula mencionada en el párrafo anterior se tiene que $7-1+1=7$ es la cantidad máxima de substrings de tamaño 1 de esta palabra. Recorriendo el arreglo LCP es necesario encontrar aquellos valores que sean \textbf{mayores o iguales que $k$} para restarlos a la cantidad máxima de diferentes substrings, porque ese valor indica en el arreglo de sufijos si determinado sufijo se \textbf{repite más de una vez}, por consiguiente esto indica que disminuye en una unidad la cantidad total de diferentes substrings de determinado tamaño. Aplicando en el caso anterior:

Máxima cantidad de diferentes substrings de tamaño 1 para BANANA\$: $DS = 7$

$a)$ $LCP[0]=0 =>$ DS se mantiene $=> DS=7$\\
$b)$ $LCP[1]=0 =>$ DS se mantiene $=> DS=7$\\ 
$c)$ $LCP[2]=1 =>$ $DS=7-1=6$\\ 
$d)$ $LCP[3]=3 =>$ $DS=6-1=5$\\
$e)$ $LCP[4]=0 =>$ DS se mantiene $=> DS=5$\\
$f)$ $LCP[5]=0 =>$ DS se mantiene $=> DS=5$\\
$g)$ $LCP[6]=2 =>$ $DS=5-1=4$ 

Diferentes substrings en total de tamaño 1 en la palabra BANANA\$: 4.

Para los tamaños 2 hasta 7 (palabra completa) los diferentes substrings encontrados son los siguientes:\\

\begin{table}[H]
\centering
\label{propuesta-12}
\begin{tabular}{|c|c|c|}
\hline
Tamaño 2     & Tamaño 3      &  Tamaño 4   \\
Substrings totales: 6      &  Substrings totales: 5    & Substrings totales: 4   \\
Elementos LCP $\geq$ 2: 2       & Elementos LCP $\geq$ 3: 1       & Elementos LCP $\geq$ 4: 0      \\
DS de tamaño 2: $6-2 = 4$      & DS de tamaño 3: $5-1 = 4$          & DS de tamaño 4: $4-0 = 4$            \\ \hline
Tamaño 5     & Tamaño 6      &  Tamaño 7   \\
Substrings totales: 3      &  Substrings totales: 2    & Substrings totales: 1   \\
Elementos LCP $\geq$ 5: 0       & Elementos LCP $\geq$ 6: 0       & Elementos LCP $\geq$ 7: 0      \\
DS de tamaño 5: $3-0 = 3$      & DS de tamaño 6: $2-0 = 2$          & DS de tamaño 7: $1-0 = 1$            \\ \hline
\end{tabular}
\end{table}

Sumando los DS encontrados entre los tamaños 1 hasta 7 el valor es de $4+4+4+4+3+2+1=22$, obteniendo el mismo valor de antes.

\subsection{Aumentando la cantidad de diferentes substrings desde 0 considerando determinados tamaños de LCPs consecutivos}

Esta segunda perspectiva considera recorrer el arreglo LCP con una pequeña variación, la que sería mover el primer elemento del arreglo (que siempre será 0) y dejarlo en la última posición desde izquierda a derecha:

\begin{table}[h]
\centering
\label{propuesta-2}
\begin{tabular}{|l|l|l|l|l|l|l|l|l|l|l|l|l|l|l|}
\cline{1-7} \cline{9-15}
0 & 0 & 1 & 3 & 0 & 0 & 2 & -\textgreater & 0 & 1 & 3 & 0 & 0 & 2 & 0 \\ \cline{1-7} \cline{9-15} 
\end{tabular}
\caption{Arreglo LCP (tabla izquierda) mueve su primer elemento (0) hacia la última posición (tabla derecha).}
\end{table}

El motivo de esto es identificar el prefijo común más largo entre los 2 sufijos consecutivos entre los sufijos $x$ e $y$ considerando al {\textbf{primero o sufijo \textit{x}}} como el sufijo de referencia:

\begin{table}[H]
	\centering
	\label{propuesta-21}
	\begin{tabular}{c c l}
		\textit{\textbf{SA[]}} & \textit{\textbf{LCP[]}} &\textit{\textbf{sufijo}}\\
		6 & 0 & \$\\
		5 & 1 & A\$\\
		3 & 3 & ANA\$\\
		1 & 0 & ANANA\$\\
		0 & 0 & BANANA\$\\
		4 & 2 & NA\$\\
		2 & 0 & NANA\$\\
	\end{tabular}
\caption{SA y arreglo LCP modificado de la palabra BANANA\$}
\end{table}

Para que sea más entendible, el último valor del arreglo LCP modificado es 0 ya que NANA\$ es el último sufijo, y no tiene un sufijo posterior con el cual compararse.

En esta ocasión no se considerará la máxima cantidad de diferentes substrings de tamaño $k$ ($n-k+1$) ya que todo se obtendrá del \textit{suffix array} y de su arreglo LCP correspondiente, y el realizar esta modificación en el arreglo LCP permitirá lograr el segundo objetivo para este trabajo, que es el de encontrar a los conjuntos de péptidos que más se repiten para un determinado tamaño. Se puede ejemplificar esto con la búsqueda de los diferentes substrings de tamaño 3 en la palabra BANANA\$:

Se inicializa con $DS = 0$.

a) $LCP[0]=0 =>$ se mantiene $=> DS=0$ ya que el tamaño del sufijo es menor a 3 ($SA[0]=$ \$).\\
b) $LCP[1]=1 =>$ se mantiene $=> DS=0$ ya que ocurre el mismo fenómeno de antes ($SA[1]=$ A\$).\\ 
c) $LCP[2]=3 => SA[2] =$ ANA\$, este sufijo con su siguiente sufijo consecutivo tiene como \textit{longest common prefix} a ANA, por lo tanto se sabe que ANA se repite al menos 2 veces en la palabra; por ahora se seguirá dejando $DS = 0$.

Aquí viene la premisa del LCP consecutivo, ya que si el siguiente valor del arreglo LCP (para este caso $LCP[3]$) fuera mayor o igual que 3, entonces se tendría una nueva repetición del prefijo ANA, por lo tanto ahora serían 3 las veces que este prefijo estaría repetido en la palabra. Entendiéndolo de manera más formal, se tendrían \textbf{$l$ valores consecutivos desde la posición $s$ del arreglo LCP que serían mayores que $k$ (que para este caso es 3)}, entregando un total de $l+1$ repeticiones del sufijo $SA[s]$ de tamaño $k$. En caso contrario (valor del arreglo LCP menor que 3), se acabarían las repeticiones de determinado sufijo y se agrega una unidad al total de diferentes substrings encontrados.

d) $LCP[3]=0 =>$ Aquí el valor es menor que 3, por lo tanto $DS=0+1=1$ y se tiene que ``ANA'' se repite 2 veces.\\
e) $LCP[4]=0 =>$ Tamaño $SA[4] = 7$, $SA[4,3] =$ BAN, por lo tanto $DS=1+1=2$.\\
f) $LCP[5]=2 =>$ Tamaño $SA[5] = 3$, $SA[5,3] =$ NA\$, por lo tanto $DS=2+1=3$.\\
g) $LCP[6]=0 =>$ Tamaño $SA[6] = 5$, $SA[6,3] =$ NAN, por lo tanto $DS=3+1=4$.\\

Por consiguiente, se tiene que los diferentes substrings de tamaño 3 para la palabra BANANA\$ son 4, ANA que se repite 2 veces, BAN, NA\$ y NAN, que se repiten solo una vez. Sumando estas cantidades se tiene un valor de 5, que es el \textbf{número total de substrings de tamaño 3} ($n-k+1$).

Por ende para la realización del algoritmo se utilizará esta segunda premisa, considerando las restricciones pertinentes para este trabajo.

\subsection{Restricciones para la propuesta}

Primero que todo, se extraerán las secuencias de polipéptidos de los archivos .fasta y se alinearán en una \textbf{única gran cadena} donde cada secuencia estará unida por un signo \$, con esto será posible identificar en los arreglos si cierto sufijo está compuesto por este signo o no, de esa forma descartarlo dentro de los diferentes residuos de aminoácidos que se cuentan:

\begin{table}[h]
\centering
\label{propuesta-22}
\begin{tabular}{c}
$\ldots$MPSTLQVLAKKVLKENDHISR\$EYHILKCWHEAPIILCFNGSKQM$\ldots$\\ 
\end{tabular}
\caption{2 secuencias enlazadas en una cadena general utilizando el signo \$ como unión.}
\end{table}

Para esta cadena grande (de largo $m$), el arreglo de sufijos y el arreglo LCP tendrán tamaño $m$, por lo cual para determinar los diferentes substrings de tamaño $k$ y aquellos substrings que más se repiten se debe recorrer el arreglo LCP completo, considerando:

\begin{enumerate}
\item Si el largo del sufijo es mayor o igual a $k$, entonces el arreglo LCP puede ser analizado, en caso contrario se omite y se continúa al siguiente valor del arreglo LCP.
\item Si el prefijo del sufijo revisado \textbf{solamente esté compuesto por los 20 aminoácidos conocidos} \cite{biomolecula}. Otros aminoácidos que no han sido definidos, como B, J, O, U o X serán omitidos para este problema (si son parte del substring del sufijo revisado, se omitirá y se continuará al arreglo LCP siguiente) y considerados como prohibidos \cite{aminoacids}. El signo \$ también será incluído a este grupo de carácteres prohibidos.
\end{enumerate}

Con respecto a esto es posible obtener los substrings diferentes y la cantidad de repeticiones que posee cada uno de estos, para posteriormente guardarlos en algun tipo de lista o vector. El problema es tratar de acceder a aquellos substrings que más se repiten, detalle que se verá en la siguiente sección.

\section{Algoritmo desarrollado}

Para la obtención de los diferentes substrings se realizó un código implementado en lenguaje C++ siguiendo varios puntos, en primera instancia para la base de datos de SwissProt y TrEMBL se realizó una extracción previa de datos.

El archivo ``uniprot\_sprot.fasta'' está compuesto por 555426 proteínas con un peso total de 268 MB, mientras que el ``uniprot\_trembl.fasta'' está compuesto por 88032926 proteínas con un peso de 40 GB. Para ambos archivos la construcción del arreglo de sufijos y el arreglo LCP serán diferentes pero recibirán la misma cadena enlazada, cuya construcción será explicada en la siguiente sección.

\subsection{Extracción de proteínas desde el archivo .fasta}

EL archivo \textit{.fasta} entrega cada polipéptido con un código o ID (que comienza con un $>$), a continuación en la misma línea se tiene al nombre taxativo de la proteína, y en la línea siguiente viene la cadena como tal, para luego repetir el proceso:

\begin{figure}[h]
    \centering
    \includegraphics[width=0.9\textwidth]{./images/fastadefecto.png}
    \caption{Archivo \textbf{.fasta} por defecto con varias proteínas}
    \label{fig:image5}
\end{figure}

Para extraer las cadenas se implementó el siguiente código en C++:

\begin{lstlisting}[language=C++, caption=Creación de cadena de proteínas]
ifstream fin("uniprot_sprot.fasta"); //abrir archivo base
if(!fin){
	cerr << "Couldn't open the input file!";
	return(1);
}
ofstream outputfile; //crear archivo destino de cadena
outputfile.open("substrings.txt"); //definir nombre de archivo a crear
string line; 
int cantidad_ss = 1;

getline(fin, line); 
getline(fin, line); //tomar secuencia (1)

while(fin){
	if(line[0] == '>'){ // revisar primer elemento del string (2)
		outputfile << "$";
		cantidad_ss = cantidad_ss + 1;
	}
	else{
		outputfile << line;
	}
getline(fin, line); // continuar con la siguiente linea (3)
}

outputfile.close();
\end{lstlisting}

Lo que hace este código es crear un nuevo archivo ``substrings.txt'' en la variable \textbf{outputfile} donde se guardará la cadena de proteínas. Con \textit{getline} se lee la primera ĺinea del archivo .fasta que está guardada en la variable \textbf{fin}, e inmediatamente después se llama nuevamente a \textit{getline} para leer la segunda línea del archivo .fasta, que es \textbf{una secuencia o parte de ella} (1). Luego se condiciona a realizar una de las 2 tareas siempre y cuando el archivo .fasta no se haya leído completamente: 

\begin{enumerate}
\item Si el primer elemento de la línea leída del archivo .fasta es $>$, significa que se llegó al final de una secuencia, por lo tanto es el comienzo de la siguiente (es la linea de definición de la proteína), en ese caso se le agrega el signo \$ al archivo destino como separador (2). 
\item En caso contrario, se le agrega directamente la línea completa al archivo destino, ya que esta línea solamente está compuesta por \textbf{la secuencia en sí}.
\end{enumerate}

Finalmente se vuelve a llamar a \textit{getline} para continuar con la siguiente línea del archivo .fasta (3).

En resumen el formato del nuevo archivo ``substrings.txt'' es una sola línea que tiene concatenada todas las proteínas:

\begin{table}[h]
\centering
\label{propuesta-23}
\begin{tabular}{c}
TSCPGGNHPVCCSTDLCNK\$MKTL$\ldots$SDLT\$LKCNKLVPLFYKTCP\\ 
\end{tabular}
\caption{Ejemplo de cadena encontrada en el archivo destino}
\end{table}

Teniendo esta gran línea ya se puede construir el arreglo de sufijos y el arreglo LCP. Considerar como dato relevante que guardando esta cadena en una variable tipo \textit{string} cada caracter ocupa el tamaño de 1 byte de capacidad \cite{manipulatingstrings}, y esto determinará de qué manera se construirán los arreglos para la solución de este problema.

Por defecto se tiene que el tamaño de UniProt-SwissProt es de 268 MB y su cadena de proteínas tiene un peso de 199 MB, mientras que para UniProt-TrEMBL su tamaño por defecto es de 40 GB y su cadena de proteínas posee un peso de 30 GB, si se tiene en consideración que para guardar la cadena en una variable esta se almacena en la memoria RAM del ordenador. En primera instancia no hay problema con la base de datos de SwissProt ya que el espacio ocupado es suficiente si se trabaja en un computador normal (si se guardan los arreglos en una variable tipo \textit{vector}, cada elemento del vector pesa 4 bytes), pero para la cadena creada en función de UniProt-TrEMBL parece ser más complejo, ya que el string para almacenar la cadena requeriría 30 GB de memoria RAM, inclusive ocupando un servidor con 50 GB de capacidad de RAM sería inviable construir los arreglos (en cálculos sencillos se necesitaría de una RAM de al menos 300 GB para realizar esta tarea). Una posible solución para esto se revisará más adelante en la implementación para la base de datos UniProt-TrEMBL.

\subsection{Implementación solución para base de datos UniProt-SwissProt}

Para la implementación del algoritmo se trabajó en base al algoritmo de Kasai (poner cite) para obtener el arreglo LCP en base al arreglo de sufijos.

\subsubsection{Arreglo de sufijos}

%archivo implementacion
\chapter{Análisis de resultados}


\section{Resultados obtenidos}

\section{Análisis detallado}

\subsection{Análisis matemático}

\subsection{Análisis biológico}

%fin desarrollo

%archivo conclusiones
\chapter*{Conclusiones}
\addcontentsline{toc}{chapter}{Conclusiones}

%Colocar los principales descubrimientos (aunque se repitan)

Las proteínas constituyen una parte fundamental de la biología actual, son partícipes de numerosas funciones vitales en los seres vivos, al día de hoy se siguen investigando sobre nuevos polipéptidos que van apareciendo y también se siguen buscando cuáles son los aminoácidos y/o secuencias responsables directamente de estas funciones vitales. Estas proteínas han sido registradas y guardadas en archivos \textit{.fasta} como cadenas de strings, y pueden ser descargadas de manera gratuita en las páginas encargadas, como UniProt y EROP-Moscow.

Lo que se puso como objetivo en este trabajo es desarrollar en base a estructuras de indexación (Indexed Text Searching) una manera entendible y conveniente para obtener los diferentes fragmentos de proteínas de largo entre 1 hasta 50 aminoácidos, y a partir de aquellos extraer las secuencias que más se repetían. Y mirando los resultados logrados en la sección anterior, se puede concluir que el uso del \textbf{arreglo de sufijos}, \textbf{arreglo LCP} y \textbf{\textit{priority queue}} para este trabajo permitió lograr con éxito esta tarea.

\subsubsection{Problemas presentados en el desarrollo de la implementación}

Aunque en el desarrollo de la implementación, aparecieron variados problemas a la hora de trabajar ciertos aspectos. Primero fue el hecho \textbf{de sustraer las cadenas de proteínas de los archivos .fasta y alojarlas en otro archivo encadenándolas en un solo string}, donde el tiempo empleado para realizar esto dependía del tamaño del archivo analizado.\\
Un segundo problema fue que la cadena de strings obtenidas de la base de datos de UniProt-TrEMBL era muy grande para formar el vector correspondiente al arreglo de sufijos, lo cual originó \textbf{la búsqueda de una nueva alternativa de implementación} de esta estructura en conjunto con su arreglo LCP (algoritmos en memoria externa). Este hecho permitió comparar resultados logrados con los \textit{datasets} usados y de esa manera usar esta ``comparación'' como ejemplo de veracidad de los resultados obtenidos, en especial para la sección de los residuos que más se repiten.\\
El tercer problema fue el hecho de identificar en los archivos \textbf{cuáles eran los caracteres prohibidos que no se debían tomar en cuenta} para desarrollar este problema. Ante eso las bases de datos pertenecientes a UniProt (SwissProt-TrEMBL-Homosapiens) tenían diferentes letras prohibidas que la base de datos de EROP-Moscow (este archivo tenía cadenas de oligopéptidos en los que se encontraban cadenas de radicales polarizados en los extremos colocados con los signos $+$ (positivo) y $-$ (negativo)). Estos detalles hicieron modificar ciertas líneas del archivo principal agregando más letras prohibidas a revisar. En consecuencia supuso un aumento (que no fue significativo) en los tiempos al obtener los resultados finales.

\subsubsection{Extensión del algoritmo a otros temas}

Si bien el algoritmo fue desarrollado con el objetivo de hallar los diferentes fragmentos de secuencias de proteínas y sus repeticiones, también se puede tomar y modificar este algoritmo para trabajar en:

\begin{enumerate}

\item Cadenas de ADN (diferentes substrings de tamaño $k$ y cuáles son los que más se repiten).
\item Todo tipo de textos, ya sean científicos, literarios, de entretención, etc. (buscar cantidad de vocales y/o trozos diferentes de palabras de determinado tamaño que hay en un determinado texto).

\end{enumerate}

Con respecto a la opción de cadenas de ADN, existen variados genomas disponibles a descargar los cuales también se encuentran en formato .fasta. Por lo mismo manipularlos a la hora de extraer la cadena tendría el mismo procedimiento que la obtención de la cadena de proteínas, incluso es más sencilla ya que solamente se trabajaría con una cadena genérica de ADN, no habría concatenación con otras cadenas, por otra parte acá se tendrían 4 nucleótidos a revisar y la combinatoria de potenciales substrings de tamaño $k$ a encontrar sería mucho más pequeña.\\
Por el lado de los textos sería más difícil extraer una cadena de palabras, se tendrían que eliminar los espacios o considerarlos como carácteres prohibidos a igual que los elementos paraverbales, como los signos de interrogación, signos de exclamación, puntos, etc. Por otro lado, el universo de letras a considerar serían 27 (las letras del abecedario), de manera que la combinatoria de posibles secuencias de palabras sería bastante más grande que con las proteínas. Lo positivo de trabajar con estos textos es que se utilizan \textbf{caracteres ordenados en ``palabras''} por lo cual aplicando el algoritmo a estos textos y obteniendo resultados permitiría estudiar el tipo de lenguaje o expresiones que usa un determinado autor para escribir su obra.

\subsubsection{Posible trabajo a futuro}

La solución entregada en esta memoria con respecto al problema analizado de ninguna manera puede ser catalogado como definitivo, obviamente podrían haber mejores propuestas de implementaciones. El real potencial de esto es el \textbf{uso del arreglo de sufijos en una aplicación} demostrando que su utilización es muy importante para trabajar cadenas que para este caso fueron las proteínas. En el caso de los tiempos logrados pueden ser más pequeños si se utilizará un servidor más potente (el servidor de la Universidad es bastante completo a nivel de velocidad y memoria RAM) o mejorar ciertos pasos de la implementación, por ejemplo cuando \textbf{se pasan los datos de los archivos .sa y .lcp a los archivos manipulables .txt para usarlos en el programa principal}. Sería bueno intentar extraer toda esa información sin crear un archivo auxiliar, lo que no solo impondría una mejora notable en los tiempos, sino que también se ocuparía mucho menos espacio en el disco duro.

En lo relativo a investigación, actualmente se está analizando los diferentes substrings de tamaño $k$ y aquellos que más se repiten de ciertos organismos (animales, plantas, bacterias, entre otros) extraídos de las bases de datos de UniProt (SwissProt y TrEMBL) y EROP-Moscow, de los cuáles se realizarán interpretaciones biológicas y fisicoquímicas de los resultados obtenidos.

En definitiva siempre será relevante trabajar con proteínas, a cada día, a cada minuto se va descubriendo una nueva proteína, una nueva función correspondiente a un determinado aminoácido, posiblemente se lleguen a descubrir cientos de péptidos en el futuro ya que aún quedan muchas cosas por resolver con respecto a las propiedades de estas estructuras.

Y con respecto a las estructuras de indexación, sus avances y motivación de trabajo han ido a una velocidad tan rápida que a casi 20 años de su creación ya ocupan un sector importante para analizar cadenas de strings. Esto ha ido de la mano directamente con el crecimiento de las tecnologías de información (TI).




\begin{thebibliography}{9}
\bibitem{latexcompanion} 
Michel Goossens, Frank Mittelbach, and Alexander Samarin. 
\textit{The \LaTeX\ Companion}. 
Addison-Wesley, Reading, Massachusetts, 1993.
 
\bibitem{einstein} 
Albert Einstein. 
\textit{Zur Elektrodynamik bewegter K{\"o}rper}. (German) 
[\textit{On the electrodynamics of moving bodies}]. 
Annalen der Physik, 322(10):891–921, 1905.
 
\bibitem{knuthwebsite} 
Knuth: Computers and Typesetting,
\\\texttt{http://www-cs-faculty.stanford.edu/\~{}uno/abcde.html}

\bibitem{bioinformatica} 
Rafael Lahoz-Beltrá.
\textit{BIOINFORMÁTICA, simulación, vida artificial e inteligencia artificial}. 
Ediciones Díaz de Santos S.A., Madrid, España, 2004.

\bibitem{searching} 
Bruce C. Orcutt, Winona C. Barker.
\textit{Searching the protein sequence database}.
National Biomedical Research Foundation, Georgetown University Medical Center,
Washington, D.C., \textit{Bulletin of Mathematical Biology}, 46(4):545-552, 1984.

\bibitem{array} 
Alexander A. Zamyatnin.
\textit{The Features of an Array of Natural Oligopeptides}.
Bakh Institute of Biochemistry, Russian Academy of Sciences, Moskow, Russia;
Departamento de Informática, Universidad Técnica Federico Santa María, Valparaíso, Chile,
\textit{Neurochemical Journal}, 10(4):249-257, 2016.

\bibitem{biomolecula} 
Wikipedia, Definición de biomolécula,
\\\texttt{https://es.wikipedia.org/wiki/Biomolécula}.

\end{thebibliography}



%\bibliographystyle{plain}
\bibliography{bib/papers}


%\backmatter
%\cleardoublepage

\singlespacing
\cleardoublepage





\end{document}
