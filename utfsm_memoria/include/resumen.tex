\chapter*{Resumen}
\addcontentsline{toc}{chapter}{Resumen}

Esta memoria tratará sobre el uso de un tipo de estructura de indexación de textos conocido como el \textbf{arreglo de sufijos} y su derivado directo, el \textbf{arreglo LCP} para obtener de una cadena de proteínas la cantidad total de diferentes substrings de tamaño $k$ que variará entre 1 hasta 50, y para cada $k$ encontrar cuáles son los residuos que más se repiten. Para ello los archivos a utilizar para generar esta cadena son 4 que corresponden a las bases de datos de proteínas UniProt-SwissProt, UniProt-TrEMBL, EROP-Moscow y Homosapiens (extraído de UniProt-SwissProt). Se utilizarán 2 algoritmos para lograr este propósito, que construirán el arreglo de sufijos y el arreglo LCP de diferentes maneras y que tendrán en común el uso de la estructura conocida como \textit{priority queue} para guardar aquellos residuos que más se repitan.

