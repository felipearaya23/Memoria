\chapter*{Apéndice}

\begin{algorithm}
\begin{algorithmic}[1]
\STATE $n = length[T]$;
\STATE $m = length[P]$;
\STATE $F = compute$\_$prefix$\_$function(P)$;
\STATE $q = 0$; 
\FOR {$i = 1$ \textbf{to} $n$}
    \WHILE {$q > 0$ \textbf{and} $P[q+1] \neq T[i]$}
    	\STATE $q = F[q]$
	\ENDWHILE
	\IF {$P[q+1] == T[i]$}
		\STATE $q = q + 1$
	\ENDIF
	\IF {$q == m$}
		\STATE print ``patron ocurre con shift''  $i - m$
		\STATE $q = F[q]$
	\ENDIF
\ENDFOR
\end{algorithmic}
\caption{Knuth-Morris-Pratt} \label{alg:algoritmo1}
\end{algorithm}

\begin{algorithm}
\begin{algorithmic}[1]
\REQUIRE String $T$ (texto de largo $n$ caracteres) y $P$ (patron de largo $m$ caracteres)
\ENSURE Indicacion de que si $P$ es un substring de $T$, o si $P$ no es un substring de $T$
\STATE $i = m-1$
\STATE $j = m-1$
\REPEAT
	\IF {$P[j] == T[i]$}
		\IF {$j == 0$}
			\RETURN $i$
		\ELSE
			\STATE $i = i-1$
			\STATE $j = j-1$
		\ENDIF
	\ELSE
		\STATE $i = i + m - j - 1$
		\STATE $i = i + max(j-last(T[i]), match(j))$
		\STATE $j = m-1$
	\ENDIF
\UNTIL {$i > n-1$}
\RETURN ``No hay substring en $T$ que sea igual a $P$''

\end{algorithmic}
\caption{Boyer-Moore} \label{alg:algoritmo2}
\end{algorithm}
