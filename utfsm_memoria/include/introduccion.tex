\chapter*{Introducci\'on}
\addcontentsline{toc}{chapter}{Introducci\'on}

Este trabajo de se encuentra en una rama que en los últimos años ha tomado bastante importancia en la informática, conocida como la {\textit{Bioinformática}} \cite{bioinformatica}, la cual aplica las tecnologías computacionales contemporáneas a estructuras como ADN, proteínas, entre otras estructuras biológicas complejas y los cuales están a la mano del ser humano como archivos de cadenas de secuencias.

Las proteínas (compuestas de combinaciones de 20 aminoácidos) han sido objetos de estudios \cite{searching, array} con respecto a la cantidad de aminoácidos que los conforman, según el año de descubrimiento de las proteínas o el tipo de proteína.

Las combinaciones llamadas residuos (fragmentos) de aminoácidos ({\textit{amino acid residues}} o AAR de manera abreviada en inglés) se determinan según su largo $k$. Buscar ciertos residuos dada la cantidad de proteínas que existen en la actualidad es un desafío muy grande.