\chapter*{Abstract}
\addcontentsline{toc}{chapter}{Abstract}

The purpose of this thesis is to search and identify in files corresponding to the UniProt-SwissProt, UniProt-TrEMBL, EROP-Moscow and Homosapiens (extracted from UniProt-SwissProt) proteins databases, the amount of different protein fragments of long $ k $ between 1 to 50 and identify which are the most repeated fragments. This task is important because the amount of proteins that are appearing every day grows and therefore, it is interesting to consider what is the percentage of different fragments of peptides that is currently covered.

To do this, a text string generated from each database was created and based on this string, a type of text indexing structure known as the \textbf{suffix array} and its direct derivative, the \textbf{LCP array} to perform the tasks mentioned in the previous paragraph. Two algorithms will be used to achieve this purpose, which will build the suffix array and the LCP array in different ways and that will have in common the use of the structure known as \textbf{\textit {priority queue}} to store those residues that are most repeated.

The results obtained for each database have in common that as $k$ grows, the percentage of different fragments decreases and the most repeated fragments follow similar physical-chemical characteristics.