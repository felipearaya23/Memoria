\chapter*{Abstract}
\addcontentsline{toc}{chapter}{Abstract}

This thesis will deal with the use of a type of text indexing structure known as the \textbf {suffix array} and its direct derivative, the \textbf{LCP array} to obtain from a chain of proteins the total amount of different substrings of size $k$ that will vary between 1 to 50, and for each $k$ find which are the most repeated residues. For this, the files to be used to generate this chain are 4 which are the UniProt-SwissProt, UniProt-TrEMBL, EROP-Moscow and Homosapiens (extracted from UniProt-SwissProt) protein databases. Two algorithms will be used to achieve this purpose, which will build the suffix array and the LCP array in different ways and that will have in common the use of the structure known as \textit{priority queue} to store those residues that are most repeated.