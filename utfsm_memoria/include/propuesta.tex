\chapter{Implementación}

\section{Propuesta considerada}

Examinando las técnicas anteriormente revisadas, se llega al punto de que lo más factible es trabajar con las cadenas de secuencias utilizando un arreglo que las encadene una a una. Recordando los objetivos que se tienen para esta memoria, estas son:

\begin{enumerate}
\item Obtener la cantidad total de diferentes residuos de aminoácidos de tamaño $k =$ 1 hasta 50 que existen para las bases de datos de UniProt-SwissProt y UniProt-TrEMBL.
\item Encontrar para cada caso anterior cuáles son los residuos de aminoácidos que más se repiten.
\end{enumerate}

Para realizar la primera tarea, será necesario construir un \textit{suffix array} el cual será la base del arreglo LCP para realizar este objetivo. Considerando un ejemplo sencillo como la palabra BANANA\$:

\begin{table}[H]
	\centering
	\begin{tabular}{c l}
		\textit{\textbf{SA[]}} & \textit{\textbf{sufijo}}\\
		6 & \$\\
		5 & A\$\\
		3 & ANA\$\\
		1 & ANANA\$\\
		0 & BANANA\$\\
		4 & NA\$\\
		2 & NANA\$\\
	\end{tabular}
\end{table}

Se puede apreciar que los números asociados a cada sufijo ya están ordenados como si fuera un arreglo de sufijos. Es posible obtener la cantidad total de diferentes substrings que componen esta palabra utilizando el arreglo LCP de la siguiente forma. Introduciendo los 2 siguientes conceptos:

{\it{length}}('X') = Largo de caracteres de la palabra 'X'.\\
{\it{LCP}}('Y','Z') = Prefijo más largo en común ({\it{Longest Common Prefix}}) entre los substrings 'Y' y 'Z'.

Y partiendo según el orden alfabético dado anteriormente, se hace el siguiente ejercicio:

Largo primer sufijo ordenado ('\$') = 1 = $var$\\
Comienzo de pares de sufijos:
\begin{enumerate}
	\item ('\$','A\$'): $var +=$ {\it{length}}('\$A') - {\it{LCP}}('\$','\$A')\\
	$var=var+2-0 =>  1+2=3$

	\item ('A\$','ANA\$'): $var +=$ {\it{length}}('ANA\$') - {\it{LCP}}('A\$','ANA\$')\\
	$var=var+4-1 =>  3+3=6$
	
	\item ('ANA\$','ANANA\$'): $var +=$ {\it{length}}('ANANA\$') - {\it{LCP}}('ANA\$','ANANA\$')\\
	$var=var+6-3 =>  6+3=9$
	
	\item ('ANANA\$','BANANA\$'): $var +=$ {\it{length}}('BANANA\$') - {\it{LCP}}('ANANA\$','BANANA\$')\\
	$var=var+7-0 =>  9+7=16$
	
	\item ('BANANA\$','NA\$'): $var +=$ {\it{length}}('NA\$') - {\it{LCP}}('BANANA\$','NA\$')\\
	$var=var+3-0 =>  16+3=19$
	
	\item ('NA\$','NANA\$'): $var +=$ {\it{length}}('NANA\$') - {\it{LCP}}('NA\$','NANA\$')\\
	$var=var+5-2 =>  19+3=22$
	
\end{enumerate}

Cantidad de diferentes substrings que hay en BANANA\$: 22.

Lo que se hace en este caso es crear una variable y guardar la longitud del primer sufijo del SA (en este caso \$) para luego realizar una comparación entre los sufijos consecutivos $i$ y $j$ adicionando en cada caso a la variable las longitudes respectivas de los sufijos $j$, y además \textbf{se le resta el prefijo común más largo entre estos sufijos consecutivos}:

\begin{table}[H]
	\centering
	\label{propuesta-1}
	\begin{tabular}{c c l}
		\textit{\textbf{SA[]}} & \textit{\textbf{LCP[]}} &\textit{\textbf{sufijo}}\\
		6 & 0 & \$\\
		5 & 0 & A\$\\
		3 & 1 & ANA\$\\
		1 & 3 & ANANA\$\\
		0 & 0 & BANANA\$\\
		4 & 0 & NA\$\\
		2 & 2 & NANA\$\\
	\end{tabular}
\caption{SA y arreglo LCP de la palabra BANANA\$}
\end{table}

Entonces, particularizando el problema, ¿cómo sería posible obtener la cantidad total de diferentes substrings de un determinado tamaño? La clave está en el \textbf{arreglo LCP} obtenido, el cual se puede utilizar desde 2 perspectivas para realizar esta tarea.

\subsection{Restando la cantidad máxima posible de diferentes substrings}

Primero que todo hay que considerar la cantidad potencial máxima de diferentes substrings de tamaño $k$ que se pueden obtener (la fórmula es $n-k+1$ donde $n$ es el largo de la palabra) y luego se recorre el arreglo LCP utilizando el valor de $k$ como un comparador.
Por ejemplo, de la palabra BANANA\$ a simple vista se sabe que los diferentes substrings de tamaño 1 que se encuentran son 4, que son A, B, N y \$. Usando la fórmula mencionada en el párrafo anterior se tiene que $7-1+1=7$ es la cantidad máxima de substrings de tamaño 1 de esta palabra. Recorriendo el arreglo LCP es necesario encontrar aquellos valores que sean \textbf{mayores o iguales que $k$} para restarlos a la cantidad máxima de diferentes substrings, porque ese valor indica en el arreglo de sufijos si determinado sufijo se \textbf{repite más de una vez}, por consiguiente esto indica que disminuye en una unidad la cantidad total de diferentes substrings de determinado tamaño. Aplicando en el caso anterior:

Máxima cantidad de diferentes substrings de tamaño 1 para BANANA\$: $DS = 7$

$a)$ $LCP[0]=0 =>$ DS se mantiene $=> DS=7$\\
$b)$ $LCP[1]=0 =>$ DS se mantiene $=> DS=7$\\ 
$c)$ $LCP[2]=1 =>$ $DS=7-1=6$\\ 
$d)$ $LCP[3]=3 =>$ $DS=6-1=5$\\
$e)$ $LCP[4]=0 =>$ DS se mantiene $=> DS=5$\\
$f)$ $LCP[5]=0 =>$ DS se mantiene $=> DS=5$\\
$g)$ $LCP[6]=2 =>$ $DS=5-1=4$ 

Diferentes substrings en total de tamaño 1 en la palabra BANANA\$: 4.

Para los tamaños 2 hasta 7 (palabra completa) los diferentes substrings encontrados son los siguientes:\\

\begin{table}[H]
\centering
\label{propuesta-12}
\begin{tabular}{|c|c|c|}
\hline
Tamaño 2     & Tamaño 3      &  Tamaño 4   \\
Substrings totales: 6      &  Substrings totales: 5    & Substrings totales: 4   \\
Elementos LCP $\geq$ 2: 2       & Elementos LCP $\geq$ 3: 1       & Elementos LCP $\geq$ 4: 0      \\
DS de tamaño 2: $6-2 = 4$      & DS de tamaño 3: $5-1 = 4$          & DS de tamaño 4: $4-0 = 4$            \\ \hline
Tamaño 5     & Tamaño 6      &  Tamaño 7   \\
Substrings totales: 3      &  Substrings totales: 2    & Substrings totales: 1   \\
Elementos LCP $\geq$ 5: 0       & Elementos LCP $\geq$ 6: 0       & Elementos LCP $\geq$ 7: 0      \\
DS de tamaño 5: $3-0 = 3$      & DS de tamaño 6: $2-0 = 2$          & DS de tamaño 7: $1-0 = 1$            \\ \hline
\end{tabular}
\end{table}

Sumando los DS encontrados entre los tamaños 1 hasta 7 el valor es de $4+4+4+4+3+2+1=22$, obteniendo el mismo valor de antes.

\subsection{Aumentando la cantidad de diferentes substrings desde 0 considerando determinados tamaños de LCPs consecutivos}

Esta segunda perspectiva considera recorrer el arreglo LCP con una pequeña variación, la que sería mover el primer elemento del arreglo (que siempre será 0) y dejarlo en la última posición desde izquierda a derecha:

\begin{table}[h]
\centering
\label{propuesta-2}
\begin{tabular}{|l|l|l|l|l|l|l|l|l|l|l|l|l|l|l|}
\cline{1-7} \cline{9-15}
0 & 0 & 1 & 3 & 0 & 0 & 2 & -\textgreater & 0 & 1 & 3 & 0 & 0 & 2 & 0 \\ \cline{1-7} \cline{9-15} 
\end{tabular}
\caption{Arreglo LCP (tabla izquierda) mueve su primer elemento (0) hacia la última posición (tabla derecha).}
\end{table}

El motivo de esto es identificar el prefijo común más largo entre los 2 sufijos consecutivos entre los sufijos $x$ e $y$ considerando al {\textbf{primero o sufijo \textit{x}}} como el sufijo de referencia:

\begin{table}[H]
	\centering
	\label{propuesta-21}
	\begin{tabular}{c c l}
		\textit{\textbf{SA[]}} & \textit{\textbf{LCP[]}} &\textit{\textbf{sufijo}}\\
		6 & 0 & \$\\
		5 & 1 & A\$\\
		3 & 3 & ANA\$\\
		1 & 0 & ANANA\$\\
		0 & 0 & BANANA\$\\
		4 & 2 & NA\$\\
		2 & 0 & NANA\$\\
	\end{tabular}
\caption{SA y arreglo LCP modificado de la palabra BANANA\$}
\end{table}

Para que sea más entendible, el último valor del arreglo LCP modificado es 0 ya que NANA\$ es el último sufijo, y no tiene un sufijo posterior con el cual compararse.

En esta ocasión no se considerará la máxima cantidad de diferentes substrings de tamaño $k$ ($n-k+1$) ya que todo se obtendrá del \textit{suffix array} y de su arreglo LCP correspondiente, y el realizar esta modificación en el arreglo LCP permitirá lograr el segundo objetivo para este trabajo, que es el de encontrar a los conjuntos de péptidos que más se repiten para un determinado tamaño. Se puede ejemplificar esto con la búsqueda de los diferentes substrings de tamaño 3 en la palabra BANANA\$:

Se inicializa con $DS = 0$.

a) $LCP[0]=0 =>$ se mantiene $=> DS=0$ ya que el tamaño del sufijo es menor a 3 ($SA[0]=$ \$).\\
b) $LCP[1]=1 =>$ se mantiene $=> DS=0$ ya que ocurre el mismo fenómeno de antes ($SA[1]=$ A\$).\\ 
c) $LCP[2]=3 => SA[2] =$ ANA\$, este sufijo con su siguiente sufijo consecutivo tiene como \textit{longest common prefix} a ANA, por lo tanto se sabe que ANA se repite al menos 2 veces en la palabra; por ahora se seguirá dejando $DS = 0$.

Aquí viene la premisa del LCP consecutivo, ya que si el siguiente valor del arreglo LCP (para este caso $LCP[3]$) fuera mayor o igual que 3, entonces se tendría una nueva repetición del prefijo ANA, por lo tanto ahora serían 3 las veces que este prefijo estaría repetido en la palabra. Entendiéndolo de manera más formal, se tendrían \textbf{$l$ valores consecutivos desde la posición $s$ del arreglo LCP que serían mayores que $k$ (que para este caso es 3)}, entregando un total de $l+1$ repeticiones del sufijo $SA[s]$ de tamaño $k$. En caso contrario (valor del arreglo LCP menor que 3), se acabarían las repeticiones de determinado sufijo y se agrega una unidad al total de diferentes substrings encontrados.

d) $LCP[3]=0 =>$ Aquí el valor es menor que 3, por lo tanto $DS=0+1=1$ y se tiene que ``ANA'' se repite 2 veces.\\
e) $LCP[4]=0 =>$ Tamaño $SA[4] = 7$, $SA[4,3] =$ BAN, por lo tanto $DS=1+1=2$.\\
f) $LCP[5]=2 =>$ Tamaño $SA[5] = 3$, $SA[5,3] =$ NA\$, por lo tanto $DS=2+1=3$.\\
g) $LCP[6]=0 =>$ Tamaño $SA[6] = 5$, $SA[6,3] =$ NAN, por lo tanto $DS=3+1=4$.\\

Por consiguiente, se tiene que los diferentes substrings de tamaño 3 para la palabra BANANA\$ son 4, ANA que se repite 2 veces, BAN, NA\$ y NAN, que se repiten solo una vez. Sumando estas cantidades se tiene un valor de 5, que es el \textbf{número total de substrings de tamaño 3} ($n-k+1$).

Por ende para la realización del algoritmo se utilizará esta segunda premisa, considerando las restricciones pertinentes para este trabajo.

\subsection{Restricciones para la propuesta}

Primero que todo, se extraerán las secuencias de polipéptidos de los archivos .fasta y se alinearán en una \textbf{única gran cadena} donde cada secuencia estará unida por un signo \$, con esto será posible identificar en los arreglos si cierto sufijo está compuesto por este signo o no, de esa forma descartarlo dentro de los diferentes residuos de aminoácidos que se cuentan:

\begin{table}[h]
\centering
\label{propuesta-22}
\begin{tabular}{c}
$\ldots$MPSTLQVLAKKVLKENDHISR\$EYHILKCWHEAPIILCFNGSKQM$\ldots$\\ 
\end{tabular}
\caption{2 secuencias enlazadas en una cadena general utilizando el signo \$ como unión.}
\end{table}

Para esta cadena grande (de largo $m$), el arreglo de sufijos y el arreglo LCP tendrán tamaño $m$, por lo cual para determinar los diferentes substrings de tamaño $k$ y aquellos substrings que más se repiten se debe recorrer el arreglo LCP completo, considerando:

\begin{enumerate}
\item Si el largo del sufijo es mayor o igual a $k$, entonces el arreglo LCP puede ser analizado, en caso contrario se omite y se continúa al siguiente valor del arreglo LCP.
\item Si el prefijo del sufijo revisado \textbf{solamente esté compuesto por los 20 aminoácidos conocidos} \cite{biomolecula}. Otros aminoácidos que no han sido definidos, como B, J, O, U o X serán omitidos para este problema (si son parte del substring del sufijo revisado, se omitirá y se continuará al arreglo LCP siguiente) y considerados como prohibidos \cite{aminoacids}. El signo \$ también será incluído a este grupo de carácteres prohibidos.
\end{enumerate}

Con respecto a esto es posible obtener los substrings diferentes y la cantidad de repeticiones que posee cada uno de estos, para posteriormente guardarlos en algun tipo de lista o vector. El problema es tratar de acceder a aquellos substrings que más se repiten, detalle que se verá en la siguiente sección.

\section{Algoritmo desarrollado}

\subsection{Obtención de datos para UniProt-SwissProt}

Para la obtención de los diferentes substrings se realizó un código implementado en lenguaje C++ siguiendo varios puntos.

\subsubsection{Extracción de proteínas desde el archivo .fasta}

El archivo ``uniprot_sprot.fasta'' está compuesto por 555426 proteínas