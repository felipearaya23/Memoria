\chapter{Implementación}

\section{Propuesta considerada}

Examinando las técnicas anteriormente revisadas, se llega al punto de que lo más factible es trabajar con las cadenas de secuencias utilizando un arreglo que las encadene una a una. Recordando los objetivos que se tienen para esta memoria, estas son:

\begin{enumerate}
\item Obtener la cantidad total de diferentes residuos de aminoácidos de tamaño $k =$ 1 hasta 50 que existen para las bases de datos de UniProt-SwissProt y UniProt-TrEMBL.
\item Encontrar para cada caso anterior cuáles son los residuos de aminoácidos que más se repiten.
\end{enumerate}

Para realizar la primera tarea, será necesario construir un \textit{suffix array} el cual será la base del arreglo LCP para realizar este objetivo. Considerando un ejemplo sencillo como la palabra BANANA\$:

\begin{table}[H]
	\centering
	\begin{tabular}{c l}
		\textit{\textbf{SA[]}} & \textit{\textbf{sufijo}}\\
		6 & \$\\
		5 & A\$\\
		3 & ANA\$\\
		1 & ANANA\$\\
		0 & BANANA\$\\
		4 & NA\$\\
		2 & NANA\$\\
	\end{tabular}
\end{table}

Se puede apreciar que los números asociados a cada sufijo ya están ordenados como si fuera un arreglo de sufijos. Es posible obtener la cantidad total de diferentes substrings que componen esta palabra utilizando el arreglo LCP de la siguiente forma. Introduciendo los 2 siguientes conceptos:

{\it{length}}('X') = Largo de caracteres de la palabra 'X'.\\
{\it{LCP}}('Y','Z') = Prefijo más largo en común ({\it{Longest Common Prefix}}) entre los substrings 'Y' y 'Z'.

Y partiendo según el orden alfabético dado anteriormente, se hace el siguiente ejercicio:

Largo primer sufijo ordenado ('\$') = 1 = $var$\\
Comienzo de pares de sufijos:
\begin{enumerate}
	\item ('\$','A\$'): $var +=$ {\it{length}}('\$A') - {\it{LCP}}('\$','\$A')\\
	$var=var+2-0 =>  1+2=3$

	\item ('A\$','ANA\$'): $var +=$ {\it{length}}('ANA\$') - {\it{LCP}}('A\$','ANA\$')\\
	$var=var+4-1 =>  3+3=6$
	
	\item ('ANA\$','ANANA\$'): $var +=$ {\it{length}}('ANANA\$') - {\it{LCP}}('ANA\$','ANANA\$')\\
	$var=var+6-3 =>  6+3=9$
	
	\item ('ANANA\$','BANANA\$'): $var +=$ {\it{length}}('BANANA\$') - {\it{LCP}}('ANANA\$','BANANA\$')\\
	$var=var+7-0 =>  9+7=16$
	
	\item ('BANANA\$','NA\$'): $var +=$ {\it{length}}('NA\$') - {\it{LCP}}('BANANA\$','NA\$')\\
	$var=var+3-0 =>  16+3=19$
	
	\item ('NA\$','NANA\$'): $var +=$ {\it{length}}('NANA\$') - {\it{LCP}}('NA\$','NANA\$')\\
	$var=var+5-2 =>  19+3=22$
	
\end{enumerate}

Cantidad de diferentes substrings que hay en BANANA\$: 22.

Lo que se hace en este caso es crear una variable y guardar la longitud del primer sufijo del SA (en este caso \$) para luego realizar una comparación entre los sufijos consecutivos $i$ y $j$ adicionando en cada caso a la variable las longitudes respectivas de los sufijos $j$, y además \textbf{se le resta el prefijo común más largo entre estos sufijos consecutivos}:

\begin{table}[H]
	\centering
	\label{propuesta-1}
	\begin{tabular}{c c l}
		\textit{\textbf{SA[]}} & \textit{\textbf{LCP[]}} &\textit{\textbf{sufijo}}\\
		6 & 0 & \$\\
		5 & 0 & A\$\\
		3 & 1 & ANA\$\\
		1 & 3 & ANANA\$\\
		0 & 0 & BANANA\$\\
		4 & 0 & NA\$\\
		2 & 2 & NANA\$\\
	\end{tabular}
\caption{SA y arreglo LCP de la palabra BANANA\$}
\end{table}

Entonces, particularizando el problema, ¿cómo sería posible obtener la cantidad total de diferentes substrings de un determinado tamaño? La clave está en el \textbf{arreglo LCP} obtenido, el cual se puede utilizar desde 2 perspectivas para realizar esta tarea.

\subsection{Restando la cantidad máxima posible de diferentes substrings}

Primero que todo hay que considerar la cantidad potencial máxima de diferentes substrings de tamaño $k$ que se pueden obtener (la fórmula es $n-k+1$ donde $n$ es el largo de la palabra) y luego se recorre el arreglo LCP utilizando el valor de $k$ como un comparador.
Por ejemplo, de la palabra BANANA\$ a simple vista se sabe que los diferentes substrings de tamaño 1 que se encuentran son 4, que son A, B, N y \$. Usando la fórmula mencionada en el párrafo anterior se tiene que $7-1+1=7$ es la cantidad máxima de substrings de tamaño 1 de esta palabra. Recorriendo el arreglo LCP es necesario encontrar aquellos valores que sean \textbf{mayores o iguales que $k$} para restarlos a la cantidad máxima de diferentes substrings, porque ese valor indica en el arreglo de sufijos si determinado sufijo se \textbf{repite más de una vez}, por consiguiente esto indica que disminuye en una unidad la cantidad total de diferentes substrings de determinado tamaño. Aplicando en el caso anterior:

Máxima cantidad de diferentes substrings de tamaño 1 para BANANA\$: $DS = 7$

a) $LCP[0]=0 =>$ DS se mantiene $=> DS=7$\\
b) $LCP[1]=0 =>$ DS se mantiene $=> DS=7$\\ 
c) $LCP[2]=1 =>$ $DS=7-1=6$\\ 
d) $LCP[3]=3 =>$ $DS=6-1=5$\\
e) $LCP[4]=0 =>$ DS se mantiene $=> DS=5$\\
f) $LCP[5]=0 =>$ DS se mantiene $=> DS=5$\\
g) $LCP[6]=2 =>$ $DS=5-1=4$ 

Diferentes substrings en total de tamaño 1 en la palabra BANANA\$: 4.

\subsection{Aumentando la cantidad de diferentes substrings desde 0 considerando determinados tamaños de LCPs consecutivos}

\section{Algoritmo desarrollado}
