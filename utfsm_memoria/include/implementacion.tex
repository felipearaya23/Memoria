\chapter{Resultados y análisis}

Aquí se mostrarán los resultados que se obtuvieron con los algoritmos mencionados en la sección anterior. Haciendo un pequeño recuento, los \textit{datasets} usados son archivos que utilizan un formato \textbf{.fasta} y son los siguientes:

\begin{table}[h]
\centering
\label{my-labelr1}
\begin{tabular}{|l|c|c|c|}
\hline
\multicolumn{1}{|c|}{\textbf{Archivo}} & \textbf{Proteínas} & \textbf{Tamaño (MB)} & \textbf{Tamaño cadena (MB)} \\ \hline
Base de datos SwissProt    & 555426                & 268.3                & 199.5              \\
Base de datos TrEMBL        & 89396316              & 40900 (aprox.)       & 30200 (aprox.)     \\
Base de datos EROP-Moscow        & 14785                 & 1.5                  & 1.1                \\
Proteínas humanas     & 86298                 & 47.2                 & 38.2               \\ \hline
\end{tabular}
\caption{\textit{Datasets} utilizados para la obtención de resultados}
\end{table} 

\section{Sistema utilizado y método de experimentación}

El trabajo se realizó en un ordenador \textit{Acer Aspire ES14}, no obstante las implementaciones se ejecutaron en un servidor prestado para la ocasión por la Universidad. Los datos relevantes a esta máquina se aprecian en el \textbf{Cuadro 4.2}.

\begin{table}[h]
\centering
\label{my-labelr2}
\begin{tabular}{|c|c|}
\hline
\textbf{Componente}     & \textbf{Descripción}                         \\ \hline
Nombre CPU              & Intel(R) Xeon(R) CPU E5-2630 @2.30GHz        \\
CPU (s)                 & 12                                           \\
Caché (capacidad)       & 15 MB                                        \\
Memoria RAM (capacidad) & 60 GB                                        \\
Sistema Operativo       & Ubuntu Server 16.04 (Linux versión 4.4.0-51) \\ \hline
\end{tabular}
\caption{Especificaciones del sistema utilizado}
\end{table}

Los códigos fueron realizados en C++, mediante el uso del entorno de desarrollo integrado (IDE) ``Code::Blocks'' \cite{codeblocks}. Al momento de ejecutar las implementaciones todo fue realizado en el terminal con los comandos básicos (asumiendo como ejemplo que el archivo base es ``prueba.cpp''):

$a)$ \texttt{g++ prueba.cpp -o pruebas}: Una vez que se compila el archivo base se crea el ejecutable ``pruebas''.\\
$b)$ \texttt{./pruebas}: Se activa el ejecutable y comienza a funcionar el código.

Los resultados se guardaron en archivos .txt, y se mostrarán en la siguiente sección.

\section{Resultados obtenidos}

\subsection{Tiempos obtenidos}

Aquí se adjuntan las tablas con los tiempos logrados con las ejecuciones de los algoritmos con las 2 implementaciones explicadas en el capítulo anterior. Siguiendo este esquema se mostrarán en una tabla los tiempos obtenidos utilizando el algoritmo ``SwissProt'' (primera implementación explicada en el capítulo anterior) y en otra tabla los tiempos otilizando el algoritmo ``TrEMBL'' (capítulo anterior, segunda implementación) ajustando la cantidad de memoria RAM usada a 20 GB para tanto el arreglo de sufijos como el arreglo LCP en memoria externa. 

En cada tabla se mostrará el tiempo en construir el arreglo de sufijos, el arreglo LCP y el tiempo empleado para obtener los diferentes substrings y los residuos que más se repiten con k entre 1 a 50 (``Tiempo programa'').

\begin{table}[h]
\centering
\label{my-labelr3}
\begin{tabular}{|c|c|c|c|}
\hline
\textbf{Archivo}  & \textbf{Tiempo SA (s)} & \textbf{Tiempo LCP (s)} & \textbf{Tiempo programa (s)} \\ \hline
SwissProt         & 2637.18                & 35.09                   & 2758.82                         \\
TrEMBL            & -                & -                &        -                  \\
EROP-Moscow       & 1.92                   & 0.03                    & 3.58                         \\
Proteínas humanas & 422.49                 & 9.06                    & 3.78 (k=1)                         \\ \hline
\end{tabular}
\caption{Tiempos de las implementaciones realizadas con los 4 \textit{datasets} utilizando el algoritmo ``SwissProt''.}
\end{table}

\begin{table}[h]
\centering
\label{my-labelr4}
\begin{tabular}{|c|c|c|c|}
\hline
\textbf{Archivo}  & \textbf{Tiempo SA (s)} & \textbf{Tiempo LCP (s)} & \textbf{Tiempo programa (s)} \\ \hline
SwissProt (EM)        &  17.98               &                    &                          \\
TrEMBL (EM)           &                 &                 &                          \\
EROP-Moscow (EM)      & 0.23                   &                     &                          \\
Proteínas humanas (EM) & 4.10                 &                     &                          \\ \hline
\end{tabular}
\caption{Tiempos de las implementaciones realizadas con los 4 \textit{datasets} utilizando el algoritmo ``TrEMBL'' (EM = \textit{External Memory}).}
\end{table}

\section{Análisis detallado}

\subsection{Análisis matemático}

\subsection{Análisis biológico}
